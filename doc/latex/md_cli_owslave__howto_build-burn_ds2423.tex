( osboxes\-:$\sim$/sketchbook/cli/owslave\$ sudo apt-\/get install gcc-\/avr\-\_\-4.\-5.\-3 )

--- using A\-Ttiny84 (attiny84) ---

-\/--- programmer -\/---

Use Arduino Uno and my \char`\"{}\-Programmer Shield\char`\"{}.

Open Arduino I\-D\-E (e.\-g. 1.\-6.\-5) and open File $>$ Examples $>$ Ardino\-I\-S\-P

Upload the sketch to an Arduino Uno board.

Adopt \char`\"{}\-Programmer Shield\char`\"{} to be wired for a tiny84, see sketch and\-: \href{http://ww1.microchip.com/downloads/en/DeviceDoc/8006S.pdf}{\tt http\-://ww1.\-microchip.\-com/downloads/en/\-Device\-Doc/8006\-S.\-pdf} \href{http://www.3bm.de/2013/09/17/attiny84-mit-arduino-uno-programmieren/}{\tt http\-://www.\-3bm.\-de/2013/09/17/attiny84-\/mit-\/arduino-\/uno-\/programmieren/}


\begin{DoxyItemize}
\item Arduino 5\-V an A\-Ttiny84 Pin 1 (V\-C\-C +)
\item Arduino G\-N\-D an A\-T\-Tiny84 Pin 14 (G\-N\-D -\/)
\item Arduino Pin 10 an A\-Ttiny84 Pin 4 (Reset)
\item Arduino Pin 11 an A\-T\-Tiny84 Pin 7 (Pin 6, P\-W\-M, Analog Input 7)
\item Arduino Pin 12 an A\-T\-Tiny84 Pin 8 (Pin 5, Analog Input 5, P\-W\-M, M\-I\-S\-O)
\item Arduino Pin 13 an A\-T\-Tiny84 Pin 9 (Pin 4, Analog Input 4, S\-C\-K)
\end{DoxyItemize}

-\/--- chip -\/---

create or add you project.\-cfg (project.\-cfg.\-tiny84) to the folder

osboxes\-:$\sim$/sketchbook/cli/owslave\$ nano Makefile

change the line  -\/c  -\/p  -\/\-C +\$\$\-T\-F\textbackslash{} into  -\/c  -\/p t84 -\/b 19200 -\/\-P /dev/tty\-A\-C\-M0 -\/\-C +\$\$\-T\-F\textbackslash{} save and quit

osboxes\-:$\sim$/sketchbook/cli/owslave\$ chmod +x cfg \hyperlink{Cfg_8py}{Cfg.\-py} cfg\-\_\-write gen\-\_\-eeprom \hyperlink{ordered__yaml_8py}{ordered\-\_\-yaml.\-py}

osboxes\-:$\sim$/sketchbook/cli/owslave\$ make C\-F\-G=project.\-cfg try1 make\mbox{[}1\mbox{]}\-: Entering directory `/home/osboxes/sketchbook/cli/owslave' mkdir -\/p device/try1 M\-C\-U\-:atmega88 M\-C\-U\-\_\-\-P\-R\-O\-G\-:m88 P\-R\-O\-G\-:avrisp A\-V\-R\-D\-U\-D\-E\-:sudo avrdude C\-F\-I\-L\-E\-S\-:\hyperlink{moat__backend_8c}{moat\-\_\-backend.\-c} main.\-c jmp.\-S \hyperlink{dev__data_8c}{dev\-\_\-data.\-c} config.\-o \hyperlink{timer_8c}{timer.\-c} \hyperlink{ds2423_8c}{ds2423.\-c} \hyperlink{onewire_8c}{onewire.\-c} \hyperlink{crc_8c}{crc.\-c} T\-Y\-P\-E\-:alert -\/1 count 0 config -\/1 port 0 humid 0 pid 0 temp 0 console 0 smoke 0 pwm 0 adc 0 mkdir -\/p device/try1 ./cfg project.\-cfg .hdr try1 avr-\/gcc -\/g -\/mmcu=atmega88 -\/\-Wall -\/\-Wstrict-\/prototypes -\/\-Os -\/mcall-\/prologues -\/fshort-\/enums -\/\-Idevice/try1 -\/c -\/o device/try1/moat\-\_\-backend.\-o \hyperlink{moat__backend_8c}{moat\-\_\-backend.\-c} avr-\/gcc -\/g -\/mmcu=atmega88 -\/\-Wall -\/\-Wstrict-\/prototypes -\/\-Os -\/mcall-\/prologues -\/fshort-\/enums -\/\-Idevice/try1 -\/c -\/o device/try1/main.\-o main.\-c main.\-c\-:120\-:17\-: warning\-: ‘bootseq’ defined but not used \mbox{[}-\/\-Wunused-\/variable\mbox{]} static uint16\-\_\-t bootseq {\bfseries attribute}((section(\char`\"{}.\-noinit\char`\"{}))); $^\wedge$ avr-\/gcc -\/g -\/mmcu=atmega88 -\/\-Wall -\/\-Wstrict-\/prototypes -\/\-Os -\/mcall-\/prologues -\/fshort-\/enums -\/\-Idevice/try1 -\/c -\/o device/try1/jmp.\-o jmp.\-S avr-\/gcc -\/g -\/mmcu=atmega88 -\/\-Wall -\/\-Wstrict-\/prototypes -\/\-Os -\/mcall-\/prologues -\/fshort-\/enums -\/\-Idevice/try1 -\/c -\/o device/try1/dev\-\_\-data.\-o \hyperlink{dev__data_8c}{dev\-\_\-data.\-c} set -\/e; \textbackslash{} ./gen\-\_\-eeprom device/try1/eprom.\-bin type \$(./cfg project.\-cfg .type try1); \textbackslash{} if ./gen\-\_\-eeprom device/try1/eprom.\-bin name $>$/dev/null 2$>$\&1 ; then \-: ; else \textbackslash{} ./gen\-\_\-eeprom device/try1/eprom.\-bin name try1 ; fi ; \textbackslash{} if \mbox{[} ds2423 != 0 \mbox{]} ; then \textbackslash{} if ./gen\-\_\-eeprom device/try1/eprom.\-bin owid serial $>$/dev/null 2$>$\&1 ; then \textbackslash{} S\-E\-R=\$(./gen\-\_\-eeprom device/try1/eprom.\-bin owid serial); \textbackslash{} if ./cfg project.\-cfg devices.\-try1.\-onewire\-\_\-id $>$/dev/null 2$>$\&1 ; then \textbackslash{} test \char`\"{}\$(./cfg project.\-cfg devices.\-try1.\-onewire\-\_\-id)\char`\"{} = \char`\"{}\$\-S\-E\-R\char`\"{} ; \textbackslash{} else \textbackslash{} ./cfg\-\_\-write project.\-cfg devices.\-try1.\-onewire\-\_\-id x\$\-S\-E\-R; \textbackslash{} fi; \textbackslash{} elif ./cfg project.\-cfg .nofollow devices.\-try1.\-onewire\-\_\-id $>$/dev/null 2$>$\&1 ; then \textbackslash{} S\-E\-R=\$(./cfg project.\-cfg devices.\-try1.\-onewire\-\_\-id); \textbackslash{} ./gen\-\_\-eeprom device/try1/eprom.\-bin owid type 0x\$(./cfg project.\-cfg codes.\-onewire.\-ds2423) serial \$\-S\-E\-R; \textbackslash{} else \textbackslash{} ./gen\-\_\-eeprom device/try1/eprom.\-bin owid type 0x\$(./cfg project.\-cfg codes.\-onewire.\-ds2423) serial random; \textbackslash{} ./cfg\-\_\-write project.\-cfg devices.\-try1.\-onewire\-\_\-id x\$(./gen\-\_\-eeprom device/try1/eprom.\-bin owid serial); \textbackslash{} fi; \textbackslash{} fi avr-\/objcopy -\/\-I binary -\/\-O elf32-\/avr --prefix-\/sections=.eeprom \textbackslash{} --redefine-\/sym \char`\"{}\-\_\-binary\-\_\-device\-\_\-try1\-\_\-eprom\-\_\-bin\-\_\-start=\-\_\-econfig\-\_\-start\char`\"{} \textbackslash{} --redefine-\/sym \char`\"{}\-\_\-binary\-\_\-device\-\_\-try1\-\_\-eprom\-\_\-bin\-\_\-size=\-\_\-econfig\-\_\-size\char`\"{} \textbackslash{} --redefine-\/sym \char`\"{}\-\_\-binary\-\_\-device\-\_\-try1\-\_\-eprom\-\_\-bin\-\_\-end=\-\_\-econfig\-\_\-end\char`\"{} \textbackslash{} device/try1/eprom.\-bin device/try1/config.\-o avr-\/gcc -\/g -\/mmcu=atmega88 -\/\-Wall -\/\-Wstrict-\/prototypes -\/\-Os -\/mcall-\/prologues -\/fshort-\/enums -\/\-Idevice/try1 -\/c -\/o device/try1/timer.\-o \hyperlink{timer_8c}{timer.\-c} avr-\/gcc -\/g -\/mmcu=atmega88 -\/\-Wall -\/\-Wstrict-\/prototypes -\/\-Os -\/mcall-\/prologues -\/fshort-\/enums -\/\-Idevice/try1 -\/c -\/o device/try1/ds2423.\-o \hyperlink{ds2423_8c}{ds2423.\-c} avr-\/gcc -\/g -\/mmcu=atmega88 -\/\-Wall -\/\-Wstrict-\/prototypes -\/\-Os -\/mcall-\/prologues -\/fshort-\/enums -\/\-Idevice/try1 -\/c -\/o device/try1/onewire.\-o \hyperlink{onewire_8c}{onewire.\-c} \hyperlink{onewire_8c}{onewire.\-c}\-:553\-:2\-: warning\-: \#warning \char`\"{}\-Ignore the 'appears to be a misspelled signal handler' warning\char`\"{} \mbox{[}-\/\-Wcpp\mbox{]} \#warning \char`\"{}\-Ignore the 'appears to be a misspelled signal handler' warning\char`\"{} $^\wedge$ \hyperlink{onewire_8c}{onewire.\-c}\-: In function ‘real\-\_\-\-P\-I\-N\-\_\-\-I\-N\-T’\-: \hyperlink{onewire_8c}{onewire.\-c}\-:554\-:6\-: warning\-: ‘real\-\_\-\-P\-I\-N\-\_\-\-I\-N\-T’ appears to be a misspelled signal handler \mbox{[}enabled by default\mbox{]} void \hyperlink{onewire_8c_ad685ebd23c8e96348e9924394dc696a8}{real\-\_\-\-P\-I\-N\-\_\-\-I\-N\-T(void)} \{ $^\wedge$ avr-\/gcc -\/g -\/mmcu=atmega88 -\/\-Wall -\/\-Wstrict-\/prototypes -\/\-Os -\/mcall-\/prologues -\/fshort-\/enums -\/\-Idevice/try1 -\/c -\/o device/try1/crc.\-o \hyperlink{crc_8c}{crc.\-c} avr-\/gcc -\/g -\/mmcu=atmega88 -\/\-Wall -\/\-Wstrict-\/prototypes -\/\-Os -\/mcall-\/prologues -\/fshort-\/enums -\/\-Idevice/try1 -\/o device/try1/image.\-elf -\/\-Wl,-\/\-Map,device/try1/image.\-map,--cref device/try1/moat\-\_\-backend.\-o device/try1/main.\-o device/try1/jmp.\-o device/try1/dev\-\_\-data.\-o device/try1/config.\-o device/try1/timer.\-o device/try1/ds2423.\-o device/try1/onewire.\-o device/try1/crc.\-o avr-\/objcopy -\/\-R .eeprom -\/\-O ihex device/try1/image.\-elf device/try1/image.\-hex avr-\/objcopy -\/j .eeprom --change-\/section-\/address .eeprom=0 -\/\-O ihex device/try1/image.\-elf device/try1/eprom.\-hex avr-\/objdump -\/h -\/\-S device/try1/image.\-elf $>$ device/try1/image.\-lss make\mbox{[}1\mbox{]}\-: Leaving directory `/home/osboxes/sketchbook/cli/owslave'

Press R\-E\-S\-E\-T button on \char`\"{}\-Programmer Shield\char`\"{}.

Insert tiny84 into \char`\"{}\-Programmer Shield\char`\"{} socket (connect I\-S\-P).

Press R\-E\-S\-E\-T button on \char`\"{}\-Programmer Shield\char`\"{}. L\-E\-D (Pin 8) should be off.

osboxes\-:$\sim$/sketchbook/cli/owslave\$ make C\-F\-G=project.\-cfg burn\-\_\-try1 B\-U\-R\-N try1 make\mbox{[}1\mbox{]}\-: Entering directory `/home/osboxes/sketchbook/cli/owslave' L\-F\-U\-S\-E\-:E2 H\-F\-U\-S\-E\-:D\-C E\-F\-U\-S\-E\-:00 E\-E\-P\-R\-O\-M\-:1 M\-C\-U\-:atmega88 M\-C\-U\-\_\-\-P\-R\-O\-G\-:m88 P\-R\-O\-G\-:avrisp A\-V\-R\-D\-U\-D\-E\-:sudo avrdude C\-F\-I\-L\-E\-S\-:\hyperlink{moat__backend_8c}{moat\-\_\-backend.\-c} main.\-c jmp.\-S \hyperlink{dev__data_8c}{dev\-\_\-data.\-c} config.\-o \hyperlink{timer_8c}{timer.\-c} \hyperlink{ds2423_8c}{ds2423.\-c} \hyperlink{onewire_8c}{onewire.\-c} \hyperlink{crc_8c}{crc.\-c} T\-Y\-P\-E\-:alert -\/1 humid 0 temp 0 pid 0 smoke 0 adc 0 port 0 config -\/1 pwm 0 console 0 count 0 set -\/e; \textbackslash{} ./gen\-\_\-eeprom device/try1/eprom.\-bin type \$(./cfg project.\-cfg .type try1); \textbackslash{} if ./gen\-\_\-eeprom device/try1/eprom.\-bin name $>$/dev/null 2$>$\&1 ; then \-: ; else \textbackslash{} ./gen\-\_\-eeprom device/try1/eprom.\-bin name try1 ; fi ; \textbackslash{} if \mbox{[} ds2423 != 0 \mbox{]} ; then \textbackslash{} if ./gen\-\_\-eeprom device/try1/eprom.\-bin owid serial $>$/dev/null 2$>$\&1 ; then \textbackslash{} S\-E\-R=\$(./gen\-\_\-eeprom device/try1/eprom.\-bin owid serial); \textbackslash{} if ./cfg project.\-cfg devices.\-try1.\-onewire\-\_\-id $>$/dev/null 2$>$\&1 ; then \textbackslash{} test \char`\"{}\$(./cfg project.\-cfg devices.\-try1.\-onewire\-\_\-id)\char`\"{} = \char`\"{}\$\-S\-E\-R\char`\"{} ; \textbackslash{} else \textbackslash{} ./cfg\-\_\-write project.\-cfg devices.\-try1.\-onewire\-\_\-id x\$\-S\-E\-R; \textbackslash{} fi; \textbackslash{} elif ./cfg project.\-cfg .nofollow devices.\-try1.\-onewire\-\_\-id $>$/dev/null 2$>$\&1 ; then \textbackslash{} S\-E\-R=\$(./cfg project.\-cfg devices.\-try1.\-onewire\-\_\-id); \textbackslash{} ./gen\-\_\-eeprom device/try1/eprom.\-bin owid type 0x\$(./cfg project.\-cfg codes.\-onewire.\-ds2423) serial \$\-S\-E\-R; \textbackslash{} else \textbackslash{} ./gen\-\_\-eeprom device/try1/eprom.\-bin owid type 0x\$(./cfg project.\-cfg codes.\-onewire.\-ds2423) serial random; \textbackslash{} ./cfg\-\_\-write project.\-cfg devices.\-try1.\-onewire\-\_\-id x\$(./gen\-\_\-eeprom device/try1/eprom.\-bin owid serial); \textbackslash{} fi; \textbackslash{} fi avr-\/objcopy -\/\-I binary -\/\-O elf32-\/avr --prefix-\/sections=.eeprom \textbackslash{} --redefine-\/sym \char`\"{}\-\_\-binary\-\_\-device\-\_\-try1\-\_\-eprom\-\_\-bin\-\_\-start=\-\_\-econfig\-\_\-start\char`\"{} \textbackslash{} --redefine-\/sym \char`\"{}\-\_\-binary\-\_\-device\-\_\-try1\-\_\-eprom\-\_\-bin\-\_\-size=\-\_\-econfig\-\_\-size\char`\"{} \textbackslash{} --redefine-\/sym \char`\"{}\-\_\-binary\-\_\-device\-\_\-try1\-\_\-eprom\-\_\-bin\-\_\-end=\-\_\-econfig\-\_\-end\char`\"{} \textbackslash{} device/try1/eprom.\-bin device/try1/config.\-o avr-\/gcc -\/g -\/mmcu=atmega88 -\/\-Wall -\/\-Wstrict-\/prototypes -\/\-Os -\/mcall-\/prologues -\/fshort-\/enums -\/\-Idevice/try1 -\/o device/try1/image.\-elf -\/\-Wl,-\/\-Map,device/try1/image.\-map,--cref device/try1/moat\-\_\-backend.\-o device/try1/main.\-o device/try1/jmp.\-o device/try1/dev\-\_\-data.\-o device/try1/config.\-o device/try1/timer.\-o device/try1/ds2423.\-o device/try1/onewire.\-o device/try1/crc.\-o avr-\/objcopy -\/\-R .eeprom -\/\-O ihex device/try1/image.\-elf device/try1/image.\-hex avr-\/objcopy -\/j .eeprom --change-\/section-\/address .eeprom=0 -\/\-O ihex device/try1/image.\-elf device/try1/eprom.\-hex avr-\/objdump -\/h -\/\-S device/try1/image.\-elf $>$ device/try1/image.\-lss T\-F=\$(mktemp avrdude-\/cfg.\-X\-X\-X\-X\-X); echo \char`\"{}default\-\_\-safemode = no;\char`\"{} $>$\$\-T\-F; \textbackslash{} E\-F\-U\-S\-E=00; \textbackslash{} \mbox{[} \char`\"{}\char`\"{} != \char`\"{}\char`\"{} \mbox{]} \&\& S\-E\-T\-\_\-\-E\-F\-U\-S\-E=\char`\"{}-\/\-U efuse\-:w\-:0x00\-:m\char`\"{}; \textbackslash{} sudo avrdude -\/c avrisp -\/p t84 -\/b 19200 -\/\-P /dev/tty\-A\-C\-M0 -\/\-C +\$\-T\-F\textbackslash{} -\/\-U flash\-:w\-:device/try1/image.\-hex\-:i -\/\-U eeprom\-:w\-:device/try1/eprom.\-hex\-:i \textbackslash{} -\/\-U lfuse\-:w\-:0x\-E2\-:m \textbackslash{} -\/\-U hfuse\-:w\-:0x\-D\-C\-:m \textbackslash{} \textbackslash{} ; X=\$?; rm \$\-T\-F; exit \$\-X

avrdude\-: A\-V\-R device initialized and ready to accept instructions

Reading $\vert$ \#\#\#\#\#\#\#\#\#\#\#\#\#\#\#\#\#\#\#\#\#\#\#\#\#\#\#\#\#\#\#\#\#\#\#\#\#\#\#\#\#\#\#\#\#\#\#\#\#\# $\vert$ 100\% 0.\-05s

avrdude\-: Device signature = 0x1e930c avrdude\-: N\-O\-T\-E\-: \char`\"{}flash\char`\"{} memory has been specified, an erase cycle will be performed To disable this feature, specify the -\/\-D option. avrdude\-: erasing chip avrdude\-: reading input file \char`\"{}device/try1/image.\-hex\char`\"{} avrdude\-: writing flash (3338 bytes)\-:

Writing $\vert$ \#\#\#\#\#\#\#\#\#\#\#\#\#\#\#\#\#\#\#\#\#\#\#\#\#\#\#\#\#\#\#\#\#\#\#\#\#\#\#\#\#\#\#\#\#\#\#\#\#\# $\vert$ 100\% 5.\-65s

avrdude\-: 3338 bytes of flash written avrdude\-: verifying flash memory against device/try1/image.\-hex\-: avrdude\-: load data flash data from input file device/try1/image.\-hex\-: avrdude\-: input file device/try1/image.\-hex contains 3338 bytes avrdude\-: reading on-\/chip flash data\-:

Reading $\vert$ \#\#\#\#\#\#\#\#\#\#\#\#\#\#\#\#\#\#\#\#\#\#\#\#\#\#\#\#\#\#\#\#\#\#\#\#\#\#\#\#\#\#\#\#\#\#\#\#\#\# $\vert$ 100\% 3.\-90s

avrdude\-: verifying ... avrdude\-: 3338 bytes of flash verified avrdude\-: reading input file \char`\"{}device/try1/eprom.\-hex\char`\"{} avrdude\-: writing eeprom (26 bytes)\-:

Writing $\vert$ \#\#\#\#\#\#\#\#\#\#\#\#\#\#\#\#\#\#\#\#\#\#\#\#\#\#\#\#\#\#\#\#\#\#\#\#\#\#\#\#\#\#\#\#\#\#\#\#\#\# $\vert$ 100\% 1.\-55s

avrdude\-: 26 bytes of eeprom written avrdude\-: verifying eeprom memory against device/try1/eprom.\-hex\-: avrdude\-: load data eeprom data from input file device/try1/eprom.\-hex\-: avrdude\-: input file device/try1/eprom.\-hex contains 26 bytes avrdude\-: reading on-\/chip eeprom data\-:

Reading $\vert$ \#\#\#\#\#\#\#\#\#\#\#\#\#\#\#\#\#\#\#\#\#\#\#\#\#\#\#\#\#\#\#\#\#\#\#\#\#\#\#\#\#\#\#\#\#\#\#\#\#\# $\vert$ 100\% 0.\-29s

avrdude\-: verifying ... avrdude\-: 26 bytes of eeprom verified avrdude\-: reading input file \char`\"{}0x\-E2\char`\"{} avrdude\-: writing lfuse (1 bytes)\-:

Writing $\vert$ \#\#\#\#\#\#\#\#\#\#\#\#\#\#\#\#\#\#\#\#\#\#\#\#\#\#\#\#\#\#\#\#\#\#\#\#\#\#\#\#\#\#\#\#\#\#\#\#\#\# $\vert$ 100\% 0.\-02s

avrdude\-: 1 bytes of lfuse written avrdude\-: verifying lfuse memory against 0x\-E2\-: avrdude\-: load data lfuse data from input file 0x\-E2\-: avrdude\-: input file 0x\-E2 contains 1 bytes avrdude\-: reading on-\/chip lfuse data\-:

Reading $\vert$ \#\#\#\#\#\#\#\#\#\#\#\#\#\#\#\#\#\#\#\#\#\#\#\#\#\#\#\#\#\#\#\#\#\#\#\#\#\#\#\#\#\#\#\#\#\#\#\#\#\# $\vert$ 100\% 0.\-02s

avrdude\-: verifying ... avrdude\-: 1 bytes of lfuse verified avrdude\-: reading input file \char`\"{}0x\-D\-C\char`\"{} avrdude\-: writing hfuse (1 bytes)\-:

Writing $\vert$ \#\#\#\#\#\#\#\#\#\#\#\#\#\#\#\#\#\#\#\#\#\#\#\#\#\#\#\#\#\#\#\#\#\#\#\#\#\#\#\#\#\#\#\#\#\#\#\#\#\# $\vert$ 100\% 0.\-02s

avrdude\-: 1 bytes of hfuse written avrdude\-: verifying hfuse memory against 0x\-D\-C\-: avrdude\-: load data hfuse data from input file 0x\-D\-C\-: avrdude\-: input file 0x\-D\-C contains 1 bytes avrdude\-: reading on-\/chip hfuse data\-:

Reading $\vert$ \#\#\#\#\#\#\#\#\#\#\#\#\#\#\#\#\#\#\#\#\#\#\#\#\#\#\#\#\#\#\#\#\#\#\#\#\#\#\#\#\#\#\#\#\#\#\#\#\#\# $\vert$ 100\% 0.\-02s

avrdude\-: verifying ... avrdude\-: 1 bytes of hfuse verified

avrdude done. Thank you.

make\mbox{[}1\mbox{]}\-: Leaving directory `/home/osboxes/sketchbook/cli/owslave' osboxes\-:$\sim$/sketchbook/cli/owslave\$

B\-I\-N\-G\-O! It works! \-:))

test now... (\href{https://github.com/M-o-a-T/owslave/blob/master/HOWTO.md}{\tt https\-://github.\-com/\-M-\/o-\/a-\/\-T/owslave/blob/master/\-H\-O\-W\-T\-O.\-md}) onewire\-\_\-io Choose which pin to connect your 1wire bus to. For now, only pins with dedicated interrupts (I\-N\-Tx) can be used. On an A\-Tmega168, these are pins D2 and D3. The default is I\-N\-T0. \href{http://ww1.microchip.com/downloads/en/DeviceDoc/8006S.pdf}{\tt http\-://ww1.\-microchip.\-com/downloads/en/\-Device\-Doc/8006\-S.\-pdf} I\-N\-T0 -\/$>$ Pin 5 P\-C\-I\-N\-T0 -\/$>$ Pin 13 V\-C\-C -\/$>$ Pin 1 G\-N\-D -\/$>$ Pin 14

D\-O\-E\-S N\-O\-T W\-O\-R\-K O\-N T\-H\-E B\-U\-S -\/ M\-A\-Y B\-E D\-U\-E T\-O T\-H\-E F\-A\-C\-T T\-H\-A\-T 't84' I\-S N\-O\-T R\-E\-C\-O\-G\-N\-I\-Z\-E\-D A\-N\-D I\-T C\-O\-M\-P\-I\-L\-E\-S A\-S 'm88'.

See also\-: \href{https://electronics.stackexchange.com/questions/205055/using-avrdude-to-program-attiny-via-arduino-as-isp}{\tt https\-://electronics.\-stackexchange.\-com/questions/205055/using-\/avrdude-\/to-\/program-\/attiny-\/via-\/arduino-\/as-\/isp} \href{https://forum.arduino.cc/index.php?topic=349912.0}{\tt https\-://forum.\-arduino.\-cc/index.\-php?topic=349912.\-0}

--- Issues ---

In project.\-cfg \char`\"{}defaults.\-target.\-t85\char`\"{} is recognized as attiny85 but does N\-O\-T compile.

In project.\-cfg \char`\"{}defaults.\-target.\-t84\char`\"{} is N\-O\-T recognized as attiny84 but does compile as m88 (atmega88).

--- using A\-Tmega88 (mega88) ---

\href{https://www.avrprogrammers.com/articles/atmega8-vs-atmega328}{\tt https\-://www.\-avrprogrammers.\-com/articles/atmega8-\/vs-\/atmega328}

-\/--- programmer -\/---

Use Arduino Uno and my \char`\"{}\-Programmer Shield\char`\"{}.

Open Arduino I\-D\-E (e.\-g. 1.\-6.\-5) and open File $>$ Examples $>$ Ardino\-I\-S\-P

Upload the sketch to an Arduino Uno board.

Adopt \char`\"{}\-Programmer Shield\char`\"{} to be wired for a mega88, see sketch and\-: \href{http://ww1.microchip.com/downloads/en/DeviceDoc/Atmel-2545-8-bit-AVR-Microcontroller-ATmega48-88-168_Summary.pdf}{\tt http\-://ww1.\-microchip.\-com/downloads/en/\-Device\-Doc/\-Atmel-\/2545-\/8-\/bit-\/\-A\-V\-R-\/\-Microcontroller-\/\-A\-Tmega48-\/88-\/168\-\_\-\-Summary.\-pdf}


\begin{DoxyItemize}
\item Arduino 5\-V an A\-Tmega88 Pin 7 (V\-C\-C +) und A\-Tmega88 Pin 20 A\-V\-C\-C
\item Arduino G\-N\-D an A\-Tmega88 Pin 8 (G\-N\-D -\/)
\item Arduino Pin 10 an A\-Tmega88 Pin 1 P\-C6 (P\-C\-I\-N\-T14/\-R\-E\-S\-E\-T)
\item Arduino Pin 11 an A\-Tmega88 Pin 17 P\-B3 (M\-O\-S\-I/\-O\-C2\-A/\-P\-C\-I\-N\-T3)
\item Arduino Pin 12 an A\-Tmega88 Pin 18 P\-B4 (M\-I\-S\-O/\-P\-C\-I\-N\-T4)
\item Arduino Pin 13 an A\-Tmega88 Pin 19 P\-B5 (S\-C\-K/\-P\-C\-I\-N\-T5)
\end{DoxyItemize}

-\/--- chip -\/---

create or add you project.\-cfg (project.\-cfg.\-mega88) to the folder

osboxes\-:$\sim$/sketchbook/cli/owslave\$ nano Makefile

change the line  -\/c  -\/p  -\/\-C +\$\$\-T\-F\textbackslash{} into  -\/c  -\/p  -\/b 19200 -\/\-P /dev/tty\-A\-C\-M0 -\/\-C +\$\$\-T\-F\textbackslash{} save and quit

( osboxes\-:$\sim$/sketchbook/cli/owslave\$ chmod +x cfg \hyperlink{Cfg_8py}{Cfg.\-py} cfg\-\_\-write gen\-\_\-eeprom \hyperlink{ordered__yaml_8py}{ordered\-\_\-yaml.\-py} )

osboxes\-:$\sim$/sketchbook/cli/owslave\$ make C\-F\-G=project.\-cfg try1 make\mbox{[}1\mbox{]}\-: Entering directory `/home/osboxes/sketchbook/cli/owslave' mkdir -\/p device/try1 M\-C\-U\-:atmega88 M\-C\-U\-\_\-\-P\-R\-O\-G\-:m88 P\-R\-O\-G\-:avrisp A\-V\-R\-D\-U\-D\-E\-:sudo avrdude C\-F\-I\-L\-E\-S\-:\hyperlink{moat__backend_8c}{moat\-\_\-backend.\-c} main.\-c jmp.\-S \hyperlink{dev__data_8c}{dev\-\_\-data.\-c} config.\-o \hyperlink{timer_8c}{timer.\-c} \hyperlink{ds2423_8c}{ds2423.\-c} \hyperlink{onewire_8c}{onewire.\-c} \hyperlink{crc_8c}{crc.\-c} T\-Y\-P\-E\-:port 0 temp 0 pwm 0 alert -\/1 adc 0 console 0 pid 0 config -\/1 humid 0 count 0 smoke 0 mkdir -\/p device/try1 ./cfg project.\-cfg .hdr try1 avr-\/gcc -\/g -\/mmcu=atmega88 -\/\-Wall -\/\-Wstrict-\/prototypes -\/\-Os -\/mcall-\/prologues -\/fshort-\/enums -\/\-Idevice/try1 -\/c -\/o device/try1/moat\-\_\-backend.\-o \hyperlink{moat__backend_8c}{moat\-\_\-backend.\-c} avr-\/gcc -\/g -\/mmcu=atmega88 -\/\-Wall -\/\-Wstrict-\/prototypes -\/\-Os -\/mcall-\/prologues -\/fshort-\/enums -\/\-Idevice/try1 -\/c -\/o device/try1/main.\-o main.\-c main.\-c\-:120\-:17\-: warning\-: ‘bootseq’ defined but not used \mbox{[}-\/\-Wunused-\/variable\mbox{]} static uint16\-\_\-t bootseq {\bfseries attribute}((section(\char`\"{}.\-noinit\char`\"{}))); $^\wedge$ avr-\/gcc -\/g -\/mmcu=atmega88 -\/\-Wall -\/\-Wstrict-\/prototypes -\/\-Os -\/mcall-\/prologues -\/fshort-\/enums -\/\-Idevice/try1 -\/c -\/o device/try1/jmp.\-o jmp.\-S avr-\/gcc -\/g -\/mmcu=atmega88 -\/\-Wall -\/\-Wstrict-\/prototypes -\/\-Os -\/mcall-\/prologues -\/fshort-\/enums -\/\-Idevice/try1 -\/c -\/o device/try1/dev\-\_\-data.\-o \hyperlink{dev__data_8c}{dev\-\_\-data.\-c} set -\/e; \textbackslash{} ./gen\-\_\-eeprom device/try1/eprom.\-bin type \$(./cfg project.\-cfg .type try1); \textbackslash{} if ./gen\-\_\-eeprom device/try1/eprom.\-bin name $>$/dev/null 2$>$\&1 ; then \-: ; else \textbackslash{} ./gen\-\_\-eeprom device/try1/eprom.\-bin name try1 ; fi ; \textbackslash{} if \mbox{[} ds2423 != 0 \mbox{]} ; then \textbackslash{} if ./gen\-\_\-eeprom device/try1/eprom.\-bin owid serial $>$/dev/null 2$>$\&1 ; then \textbackslash{} S\-E\-R=\$(./gen\-\_\-eeprom device/try1/eprom.\-bin owid serial); \textbackslash{} if ./cfg project.\-cfg devices.\-try1.\-onewire\-\_\-id $>$/dev/null 2$>$\&1 ; then \textbackslash{} test \char`\"{}\$(./cfg project.\-cfg devices.\-try1.\-onewire\-\_\-id)\char`\"{} = \char`\"{}\$\-S\-E\-R\char`\"{} ; \textbackslash{} else \textbackslash{} ./cfg\-\_\-write project.\-cfg devices.\-try1.\-onewire\-\_\-id x\$\-S\-E\-R; \textbackslash{} fi; \textbackslash{} elif ./cfg project.\-cfg .nofollow devices.\-try1.\-onewire\-\_\-id $>$/dev/null 2$>$\&1 ; then \textbackslash{} S\-E\-R=\$(./cfg project.\-cfg devices.\-try1.\-onewire\-\_\-id); \textbackslash{} ./gen\-\_\-eeprom device/try1/eprom.\-bin owid type 0x\$(./cfg project.\-cfg codes.\-onewire.\-ds2423) serial \$\-S\-E\-R; \textbackslash{} else \textbackslash{} ./gen\-\_\-eeprom device/try1/eprom.\-bin owid type 0x\$(./cfg project.\-cfg codes.\-onewire.\-ds2423) serial random; \textbackslash{} ./cfg\-\_\-write project.\-cfg devices.\-try1.\-onewire\-\_\-id x\$(./gen\-\_\-eeprom device/try1/eprom.\-bin owid serial); \textbackslash{} fi; \textbackslash{} fi avr-\/objcopy -\/\-I binary -\/\-O elf32-\/avr --prefix-\/sections=.eeprom \textbackslash{} --redefine-\/sym \char`\"{}\-\_\-binary\-\_\-device\-\_\-try1\-\_\-eprom\-\_\-bin\-\_\-start=\-\_\-econfig\-\_\-start\char`\"{} \textbackslash{} --redefine-\/sym \char`\"{}\-\_\-binary\-\_\-device\-\_\-try1\-\_\-eprom\-\_\-bin\-\_\-size=\-\_\-econfig\-\_\-size\char`\"{} \textbackslash{} --redefine-\/sym \char`\"{}\-\_\-binary\-\_\-device\-\_\-try1\-\_\-eprom\-\_\-bin\-\_\-end=\-\_\-econfig\-\_\-end\char`\"{} \textbackslash{} device/try1/eprom.\-bin device/try1/config.\-o avr-\/gcc -\/g -\/mmcu=atmega88 -\/\-Wall -\/\-Wstrict-\/prototypes -\/\-Os -\/mcall-\/prologues -\/fshort-\/enums -\/\-Idevice/try1 -\/c -\/o device/try1/timer.\-o \hyperlink{timer_8c}{timer.\-c} avr-\/gcc -\/g -\/mmcu=atmega88 -\/\-Wall -\/\-Wstrict-\/prototypes -\/\-Os -\/mcall-\/prologues -\/fshort-\/enums -\/\-Idevice/try1 -\/c -\/o device/try1/ds2423.\-o \hyperlink{ds2423_8c}{ds2423.\-c} avr-\/gcc -\/g -\/mmcu=atmega88 -\/\-Wall -\/\-Wstrict-\/prototypes -\/\-Os -\/mcall-\/prologues -\/fshort-\/enums -\/\-Idevice/try1 -\/c -\/o device/try1/onewire.\-o \hyperlink{onewire_8c}{onewire.\-c} \hyperlink{onewire_8c}{onewire.\-c}\-:553\-:2\-: warning\-: \#warning \char`\"{}\-Ignore the 'appears to be a misspelled signal handler' warning\char`\"{} \mbox{[}-\/\-Wcpp\mbox{]} \#warning \char`\"{}\-Ignore the 'appears to be a misspelled signal handler' warning\char`\"{} $^\wedge$ \hyperlink{onewire_8c}{onewire.\-c}\-: In function ‘real\-\_\-\-P\-I\-N\-\_\-\-I\-N\-T’\-: \hyperlink{onewire_8c}{onewire.\-c}\-:554\-:6\-: warning\-: ‘real\-\_\-\-P\-I\-N\-\_\-\-I\-N\-T’ appears to be a misspelled signal handler \mbox{[}enabled by default\mbox{]} void \hyperlink{onewire_8c_ad685ebd23c8e96348e9924394dc696a8}{real\-\_\-\-P\-I\-N\-\_\-\-I\-N\-T(void)} \{ $^\wedge$ avr-\/gcc -\/g -\/mmcu=atmega88 -\/\-Wall -\/\-Wstrict-\/prototypes -\/\-Os -\/mcall-\/prologues -\/fshort-\/enums -\/\-Idevice/try1 -\/c -\/o device/try1/crc.\-o \hyperlink{crc_8c}{crc.\-c} avr-\/gcc -\/g -\/mmcu=atmega88 -\/\-Wall -\/\-Wstrict-\/prototypes -\/\-Os -\/mcall-\/prologues -\/fshort-\/enums -\/\-Idevice/try1 -\/o device/try1/image.\-elf -\/\-Wl,-\/\-Map,device/try1/image.\-map,--cref device/try1/moat\-\_\-backend.\-o device/try1/main.\-o device/try1/jmp.\-o device/try1/dev\-\_\-data.\-o device/try1/config.\-o device/try1/timer.\-o device/try1/ds2423.\-o device/try1/onewire.\-o device/try1/crc.\-o avr-\/objcopy -\/\-R .eeprom -\/\-O ihex device/try1/image.\-elf device/try1/image.\-hex avr-\/objcopy -\/j .eeprom --change-\/section-\/address .eeprom=0 -\/\-O ihex device/try1/image.\-elf device/try1/eprom.\-hex avr-\/objdump -\/h -\/\-S device/try1/image.\-elf $>$ device/try1/image.\-lss make\mbox{[}1\mbox{]}\-: Leaving directory `/home/osboxes/sketchbook/cli/owslave'

Press R\-E\-S\-E\-T button on \char`\"{}\-Programmer Shield\char`\"{} if L\-E\-D (Pin 8) is on (should be off).

osboxes\-:$\sim$/sketchbook/cli/owslave\$ make C\-F\-G=project.\-cfg burn\-\_\-try1 B\-U\-R\-N try1 make\mbox{[}1\mbox{]}\-: Entering directory `/home/osboxes/sketchbook/cli/owslave' L\-F\-U\-S\-E\-:E2 H\-F\-U\-S\-E\-:D\-C E\-F\-U\-S\-E\-:00 E\-E\-P\-R\-O\-M\-:1 M\-C\-U\-:atmega88 M\-C\-U\-\_\-\-P\-R\-O\-G\-:m88 P\-R\-O\-G\-:avrisp A\-V\-R\-D\-U\-D\-E\-:sudo avrdude C\-F\-I\-L\-E\-S\-:\hyperlink{moat__backend_8c}{moat\-\_\-backend.\-c} main.\-c jmp.\-S \hyperlink{dev__data_8c}{dev\-\_\-data.\-c} config.\-o \hyperlink{timer_8c}{timer.\-c} \hyperlink{ds2423_8c}{ds2423.\-c} \hyperlink{onewire_8c}{onewire.\-c} \hyperlink{crc_8c}{crc.\-c} T\-Y\-P\-E\-:smoke 0 humid 0 temp 0 pwm 0 adc 0 port 0 config -\/1 alert -\/1 console 0 count 0 pid 0 set -\/e; \textbackslash{} ./gen\-\_\-eeprom device/try1/eprom.\-bin type \$(./cfg project.\-cfg .type try1); \textbackslash{} if ./gen\-\_\-eeprom device/try1/eprom.\-bin name $>$/dev/null 2$>$\&1 ; then \-: ; else \textbackslash{} ./gen\-\_\-eeprom device/try1/eprom.\-bin name try1 ; fi ; \textbackslash{} if \mbox{[} ds2423 != 0 \mbox{]} ; then \textbackslash{} if ./gen\-\_\-eeprom device/try1/eprom.\-bin owid serial $>$/dev/null 2$>$\&1 ; then \textbackslash{} S\-E\-R=\$(./gen\-\_\-eeprom device/try1/eprom.\-bin owid serial); \textbackslash{} if ./cfg project.\-cfg devices.\-try1.\-onewire\-\_\-id $>$/dev/null 2$>$\&1 ; then \textbackslash{} test \char`\"{}\$(./cfg project.\-cfg devices.\-try1.\-onewire\-\_\-id)\char`\"{} = \char`\"{}\$\-S\-E\-R\char`\"{} ; \textbackslash{} else \textbackslash{} ./cfg\-\_\-write project.\-cfg devices.\-try1.\-onewire\-\_\-id x\$\-S\-E\-R; \textbackslash{} fi; \textbackslash{} elif ./cfg project.\-cfg .nofollow devices.\-try1.\-onewire\-\_\-id $>$/dev/null 2$>$\&1 ; then \textbackslash{} S\-E\-R=\$(./cfg project.\-cfg devices.\-try1.\-onewire\-\_\-id); \textbackslash{} ./gen\-\_\-eeprom device/try1/eprom.\-bin owid type 0x\$(./cfg project.\-cfg codes.\-onewire.\-ds2423) serial \$\-S\-E\-R; \textbackslash{} else \textbackslash{} ./gen\-\_\-eeprom device/try1/eprom.\-bin owid type 0x\$(./cfg project.\-cfg codes.\-onewire.\-ds2423) serial random; \textbackslash{} ./cfg\-\_\-write project.\-cfg devices.\-try1.\-onewire\-\_\-id x\$(./gen\-\_\-eeprom device/try1/eprom.\-bin owid serial); \textbackslash{} fi; \textbackslash{} fi avr-\/objcopy -\/\-I binary -\/\-O elf32-\/avr --prefix-\/sections=.eeprom \textbackslash{} --redefine-\/sym \char`\"{}\-\_\-binary\-\_\-device\-\_\-try1\-\_\-eprom\-\_\-bin\-\_\-start=\-\_\-econfig\-\_\-start\char`\"{} \textbackslash{} --redefine-\/sym \char`\"{}\-\_\-binary\-\_\-device\-\_\-try1\-\_\-eprom\-\_\-bin\-\_\-size=\-\_\-econfig\-\_\-size\char`\"{} \textbackslash{} --redefine-\/sym \char`\"{}\-\_\-binary\-\_\-device\-\_\-try1\-\_\-eprom\-\_\-bin\-\_\-end=\-\_\-econfig\-\_\-end\char`\"{} \textbackslash{} device/try1/eprom.\-bin device/try1/config.\-o avr-\/gcc -\/g -\/mmcu=atmega88 -\/\-Wall -\/\-Wstrict-\/prototypes -\/\-Os -\/mcall-\/prologues -\/fshort-\/enums -\/\-Idevice/try1 -\/o device/try1/image.\-elf -\/\-Wl,-\/\-Map,device/try1/image.\-map,--cref device/try1/moat\-\_\-backend.\-o device/try1/main.\-o device/try1/jmp.\-o device/try1/dev\-\_\-data.\-o device/try1/config.\-o device/try1/timer.\-o device/try1/ds2423.\-o device/try1/onewire.\-o device/try1/crc.\-o avr-\/objcopy -\/\-R .eeprom -\/\-O ihex device/try1/image.\-elf device/try1/image.\-hex avr-\/objcopy -\/j .eeprom --change-\/section-\/address .eeprom=0 -\/\-O ihex device/try1/image.\-elf device/try1/eprom.\-hex avr-\/objdump -\/h -\/\-S device/try1/image.\-elf $>$ device/try1/image.\-lss T\-F=\$(mktemp avrdude-\/cfg.\-X\-X\-X\-X\-X); echo \char`\"{}default\-\_\-safemode = no;\char`\"{} $>$\$\-T\-F; \textbackslash{} E\-F\-U\-S\-E=00; \textbackslash{} \mbox{[} \char`\"{}\char`\"{} != \char`\"{}\char`\"{} \mbox{]} \&\& S\-E\-T\-\_\-\-E\-F\-U\-S\-E=\char`\"{}-\/\-U efuse\-:w\-:0x00\-:m\char`\"{}; \textbackslash{} sudo avrdude -\/c avrisp -\/p m88 -\/b 19200 -\/\-P /dev/tty\-A\-C\-M0 -\/\-C +\$\-T\-F\textbackslash{} -\/\-U flash\-:w\-:device/try1/image.\-hex\-:i -\/\-U eeprom\-:w\-:device/try1/eprom.\-hex\-:i \textbackslash{} -\/\-U lfuse\-:w\-:0x\-E2\-:m \textbackslash{} -\/\-U hfuse\-:w\-:0x\-D\-C\-:m \textbackslash{} \textbackslash{} ; X=\$?; rm \$\-T\-F; exit \$\-X \mbox{[}sudo\mbox{]} password for osboxes\-:

avrdude\-: A\-V\-R device initialized and ready to accept instructions

Reading $\vert$ \#\#\#\#\#\#\#\#\#\#\#\#\#\#\#\#\#\#\#\#\#\#\#\#\#\#\#\#\#\#\#\#\#\#\#\#\#\#\#\#\#\#\#\#\#\#\#\#\#\# $\vert$ 100\% 0.\-02s

avrdude\-: Device signature = 0x1e930a avrdude\-: N\-O\-T\-E\-: \char`\"{}flash\char`\"{} memory has been specified, an erase cycle will be performed To disable this feature, specify the -\/\-D option. avrdude\-: erasing chip avrdude\-: reading input file \char`\"{}device/try1/image.\-hex\char`\"{} avrdude\-: writing flash (3338 bytes)\-:

Writing $\vert$ \#\#\#\#\#\#\#\#\#\#\#\#\#\#\#\#\#\#\#\#\#\#\#\#\#\#\#\#\#\#\#\#\#\#\#\#\#\#\#\#\#\#\#\#\#\#\#\#\#\# $\vert$ 100\% 4.\-99s

avrdude\-: 3338 bytes of flash written avrdude\-: verifying flash memory against device/try1/image.\-hex\-: avrdude\-: load data flash data from input file device/try1/image.\-hex\-: avrdude\-: input file device/try1/image.\-hex contains 3338 bytes avrdude\-: reading on-\/chip flash data\-:

Reading $\vert$ \#\#\#\#\#\#\#\#\#\#\#\#\#\#\#\#\#\#\#\#\#\#\#\#\#\#\#\#\#\#\#\#\#\#\#\#\#\#\#\#\#\#\#\#\#\#\#\#\#\# $\vert$ 100\% 2.\-60s

avrdude\-: verifying ... avrdude\-: 3338 bytes of flash verified avrdude\-: reading input file \char`\"{}device/try1/eprom.\-hex\char`\"{} avrdude\-: writing eeprom (26 bytes)\-:

Writing $\vert$ \#\#\#\#\#\#\#\#\#\#\#\#\#\#\#\#\#\#\#\#\#\#\#\#\#\#\#\#\#\#\#\#\#\#\#\#\#\#\#\#\#\#\#\#\#\#\#\#\#\# $\vert$ 100\% 1.\-37s

avrdude\-: 26 bytes of eeprom written avrdude\-: verifying eeprom memory against device/try1/eprom.\-hex\-: avrdude\-: load data eeprom data from input file device/try1/eprom.\-hex\-: avrdude\-: input file device/try1/eprom.\-hex contains 26 bytes avrdude\-: reading on-\/chip eeprom data\-:

Reading $\vert$ \#\#\#\#\#\#\#\#\#\#\#\#\#\#\#\#\#\#\#\#\#\#\#\#\#\#\#\#\#\#\#\#\#\#\#\#\#\#\#\#\#\#\#\#\#\#\#\#\#\# $\vert$ 100\% 0.\-11s

avrdude\-: verifying ... avrdude\-: 26 bytes of eeprom verified avrdude\-: reading input file \char`\"{}0x\-E2\char`\"{} avrdude\-: writing lfuse (1 bytes)\-:

Writing $\vert$ \#\#\#\#\#\#\#\#\#\#\#\#\#\#\#\#\#\#\#\#\#\#\#\#\#\#\#\#\#\#\#\#\#\#\#\#\#\#\#\#\#\#\#\#\#\#\#\#\#\# $\vert$ 100\% 0.\-02s

avrdude\-: 1 bytes of lfuse written avrdude\-: verifying lfuse memory against 0x\-E2\-: avrdude\-: load data lfuse data from input file 0x\-E2\-: avrdude\-: input file 0x\-E2 contains 1 bytes avrdude\-: reading on-\/chip lfuse data\-:

Reading $\vert$ \#\#\#\#\#\#\#\#\#\#\#\#\#\#\#\#\#\#\#\#\#\#\#\#\#\#\#\#\#\#\#\#\#\#\#\#\#\#\#\#\#\#\#\#\#\#\#\#\#\# $\vert$ 100\% 0.\-01s

avrdude\-: verifying ... avrdude\-: 1 bytes of lfuse verified avrdude\-: reading input file \char`\"{}0x\-D\-C\char`\"{} avrdude\-: writing hfuse (1 bytes)\-:

Writing $\vert$ \#\#\#\#\#\#\#\#\#\#\#\#\#\#\#\#\#\#\#\#\#\#\#\#\#\#\#\#\#\#\#\#\#\#\#\#\#\#\#\#\#\#\#\#\#\#\#\#\#\# $\vert$ 100\% 0.\-02s

avrdude\-: 1 bytes of hfuse written avrdude\-: verifying hfuse memory against 0x\-D\-C\-: avrdude\-: load data hfuse data from input file 0x\-D\-C\-: avrdude\-: input file 0x\-D\-C contains 1 bytes avrdude\-: reading on-\/chip hfuse data\-:

Reading $\vert$ \#\#\#\#\#\#\#\#\#\#\#\#\#\#\#\#\#\#\#\#\#\#\#\#\#\#\#\#\#\#\#\#\#\#\#\#\#\#\#\#\#\#\#\#\#\#\#\#\#\# $\vert$ 100\% 0.\-01s

avrdude\-: verifying ... avrdude\-: 1 bytes of hfuse verified

avrdude done. Thank you.

make\mbox{[}1\mbox{]}\-: Leaving directory `/home/osboxes/sketchbook/cli/owslave' osboxes\-:$\sim$/sketchbook/cli/owslave\$

Test the device by connecting it to a 1-\/wire bus and check for the presence of a new D\-S2423.

(\href{https://github.com/M-o-a-T/owslave/blob/master/HOWTO.md}{\tt https\-://github.\-com/\-M-\/o-\/a-\/\-T/owslave/blob/master/\-H\-O\-W\-T\-O.\-md}) onewire\-\_\-io Choose which pin to connect your 1wire bus to. For now, only pins with dedicated interrupts (I\-N\-Tx) can be used. On an A\-Tmega168, these are pins D2 and D3. The default is I\-N\-T0.


\begin{DoxyItemize}
\item D\-A\-T\-A to A\-Tmega88 Pin 4 P\-D2 (P\-C\-I\-N\-T18/\-I\-N\-T0)
\item 5\-V to A\-Tmega88 Pin 7 (V\-C\-C +)
\item G\-N\-D to A\-Tmega88 Pin 8 (G\-N\-D -\/)
\item 5\-V to A\-Tmega88 Pin 20 A\-V\-C\-C
\item I\-N\-\_\-\-A to A\-Tmega88 Pin 23 P\-C0 (A\-D\-C0/\-P\-C\-I\-N\-T8)
\item I\-N\-\_\-\-B to A\-Tmega88 Pin 24 P\-C1 (A\-D\-C1/\-P\-C\-I\-N\-T9)
\end{DoxyItemize}

B\-I\-N\-G\-O! It works! \-:))

--- using Tiny (not working yet) ---

Issues with A\-Ttiny85\-: \href{https://github.com/M-o-a-T/owslave/issues/16}{\tt https\-://github.\-com/\-M-\/o-\/a-\/\-T/owslave/issues/16} Support for A\-Ttiny84\-: \href{https://github.com/M-o-a-T/owslave/issues/17}{\tt https\-://github.\-com/\-M-\/o-\/a-\/\-T/owslave/issues/17}

--- using Arduino Uno or Nano Board (not recommened) ---

You need to bypass the bootloader and program the raw/whole chip by connecting another Arduino Board to the I\-S\-P header of the first (the one to be programmed). See \href{https://www.arduino.cc/en/Tutorial/ArduinoISP}{\tt https\-://www.\-arduino.\-cc/en/\-Tutorial/\-Arduino\-I\-S\-P}

During this process the bootloader get bypassed and overwritten, so to use the Board as usual again you have to load the bootloader by I\-S\-P header again.

You should also check first whether the pins wirde on the Board are the ones you need and that they are wired properly.

The Makefile should then may be contain something like\-:  -\/c arduino -\/p atmega328p -\/b 57600 -\/\-P /dev/tty\-U\-S\-B0 -\/\-C +\$\$\-T\-F\textbackslash{} 