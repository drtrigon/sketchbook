Call the P\-J\-O\-N C++ library directly from Python 2 or Python 3 (via \href{http://cython.org/}{\tt Cython})

P\-J\-O\-N (Github\-: \href{https://github.com/gioblu/PJON/}{\tt P\-J\-O\-N} ) is an open-\/source, multi-\/master, multi-\/media (one-\/wire, two-\/wires, radio) communication protocol available for various platforms (Arduino/\-A\-V\-R, E\-S\-P8266, Teensy).

P\-J\-O\-N is one of very few open-\/source implementations of multi-\/master communication protocols for microcontrollers.

\subsection*{P\-J\-O\-N-\/cython vs P\-J\-O\-N-\/python}

{\bfseries P\-J\-O\-N-\/cython} allows you to use the C++ P\-J\-O\-N library from Python via Cython (C++ wrappers for Python) while {\bfseries P\-J\-O\-N-\/python} is a re-\/implementation of the P\-J\-O\-N protocol in Python

\subsection*{Current status\-:}

Support for P\-J\-O\-N 11.\-1 {\itshape only} and the following strategies \-:-\/
\begin{DoxyItemize}
\item Local\-U\-D\-P
\item Global\-U\-D\-P
\item Through\-Serial
\item Through\-Serial\-Async {\itshape new}
\end{DoxyItemize}

Note


\begin{DoxyItemize}
\item P\-J\-O\-N-\/cython versions are aligned with P\-J\-O\-N versions to indicate compatibility with C implementation for u\-C platforms.
\end{DoxyItemize}

\paragraph*{Python support}

Python 2.\-7, 3.\-4, 3.\-5 and 3.\-6 are tested and considered supported

\paragraph*{Platform support}

Linux and Mac O\-S X are considered supported. Windows is not supported (sorry!).

\subsection*{Install from pip}

Current version is 11.\-1.\-5-\/1

```bash pip install pjon-\/cython ```

\subsection*{Testing}

```bash \$(which python) \hyperlink{setup_8py}{setup.\-py} nosetests --with-\/doctest --doctest-\/extension=md ```

\subsection*{Global\-U\-D\-P example}

```python \begin{quotation}
\begin{quotation}
\begin{quotation}
import \hyperlink{namespacepjon__cython}{pjon\-\_\-cython} as P\-J\-O\-N class Global\-U\-D\-P(P\-J\-O\-N.\-Global\-U\-D\-P)\-:

\end{quotation}


\end{quotation}


\end{quotation}
... \# you can overload {\bfseries init} if you want ... def {\bfseries init}(self, device\-\_\-id)\-: ... P\-J\-O\-N.\-Global\-U\-D\-P.\-\_\-\-\_\-init\-\_\-\-\_\-(self, device\-\_\-id) ... self.\-packets\-\_\-received = 0 ... def receive(self, data, length, packet\-\_\-info)\-: ... print (\char`\"{}\-Recv (\{\})\-: \{\}\char`\"{}.format(length, data)) ... print (packet\-\_\-info) ... self.\-packets\-\_\-received += 1 ... self.\-reply(b'P')

\begin{quotation}
\begin{quotation}
\begin{quotation}
g = Global\-U\-D\-P(44) idx = g.\-send(123, b'H\-E\-L\-O') \section*{calling loop calls the P\-J\-O\-N bus.\-update() and bus.\-receive()}

\section*{and the return is the results of those functions -\/}

packets\-\_\-to\-\_\-send, receive\-\_\-status = g.\-loop() \section*{packets\-\_\-to\-\_\-send is the Number of packets in the P\-J\-O\-N buffer}

packets\-\_\-to\-\_\-send

\end{quotation}


\end{quotation}


\end{quotation}
1 \begin{quotation}
\begin{quotation}
\begin{quotation}
\#\-P\-J\-O\-N constants are available too receive\-\_\-status == P\-J\-O\-N.\-P\-J\-O\-N\-\_\-\-F\-A\-I\-L

\end{quotation}


\end{quotation}


\end{quotation}
True \begin{quotation}
\begin{quotation}
\begin{quotation}
\section*{When you're done with your P\-J\-O\-N interface, you can cleanup the connection by deleting it}

del g

\end{quotation}


\end{quotation}


\end{quotation}


```

\subsection*{Through Serial example}

```python \begin{quotation}
\begin{quotation}
\begin{quotation}
import \hyperlink{namespacepjon__cython}{pjon\-\_\-cython} as P\-J\-O\-N \#\-Through\-Serial Example \section*{Make sure you set self.\-bus.\-set\-\_\-synchronous\-\_\-acknowledge(false) on the other side}



class Through\-Serial(P\-J\-O\-N.\-Through\-Serial)\-:

\end{quotation}


\end{quotation}


\end{quotation}
... ... def receive(self, data, length, packet\-\_\-info)\-: ... if data.\-startswith(b'H')\-: ... print (\char`\"{}\-Recv (\{\})\-: \{\} -\/ R\-E\-P\-L\-Y\-I\-N\-G\char`\"{}.format(length, data)) ... self.\-reply(b'B\-O\-N\-Z\-A') ... else\-: ... print (\char`\"{}\-Recv (\{\})\-: \{\}\char`\"{}.format(length, data)) ... print ('') ... \begin{quotation}
\begin{quotation}
\begin{quotation}
\section*{Put your actual serial device in here...}

ts = Through\-Serial(44, b\char`\"{}/dev/null\char`\"{}, 115200) \section*{Send returns the packet's index in the packet buffer}

ts.\-send(100, b'P\-I\-N\-G 1')

\end{quotation}


\end{quotation}


\end{quotation}
0 \begin{quotation}
\begin{quotation}
\begin{quotation}
ts.\-send(100, b'P\-I\-N\-G 2')

\end{quotation}


\end{quotation}


\end{quotation}
1 \begin{quotation}
\begin{quotation}
\begin{quotation}
\section*{Error handling happens through exceptions such as P\-J\-O\-N.\-P\-J\-O\-N\-\_\-\-Connection\-\_\-\-Lost}

while True\-:

\end{quotation}


\end{quotation}


\end{quotation}
... packets\-\_\-to\-\_\-send, receive\-\_\-status = ts.\-loop() Traceback (most recent call last)\-: ... P\-J\-O\-N\-\_\-\-Connection\-\_\-\-Lost

```

\subsection*{Setting configurable properties}

```python \begin{quotation}
\begin{quotation}
\begin{quotation}
import \hyperlink{namespacepjon__cython}{pjon\-\_\-cython} as P\-J\-O\-N class Global\-U\-D\-P(P\-J\-O\-N.\-Global\-U\-D\-P)\-:

\end{quotation}


\end{quotation}


\end{quotation}
... def receive(self, data, length, packet\-\_\-info)\-: ... print (\char`\"{}\-Recv (\{\})\-: \{\}\char`\"{}.format(length, data))

\begin{quotation}
\begin{quotation}
\begin{quotation}
\section*{Global\-U\-D\-P and Local\-U\-D\-P both support set\-\_\-port to configure their U\-D\-P listening port}

g = Global\-U\-D\-P(99, 8821) del g \#\-They return the class object, so you can \char`\"{}chain them\char`\"{} pjon = Global\-U\-D\-P(100,8821).set\-\_\-autoregistration(\-False) pjon \# doctest\-: +\-E\-L\-L\-I\-P\-S\-I\-S

\end{quotation}


\end{quotation}


\end{quotation}
$<${\bfseries main}.Global\-U\-D\-P object at 0x...$>$ \begin{quotation}
\begin{quotation}
\begin{quotation}


\section*{These options affect packet overhead (in bytes)}

pjon.\-packet\-\_\-overhead()

\end{quotation}


\end{quotation}


\end{quotation}
6 \begin{quotation}
\begin{quotation}
\begin{quotation}
pjon.\-set\-\_\-crc\-\_\-32(\-True).packet\-\_\-overhead()

\end{quotation}


\end{quotation}


\end{quotation}
9 \begin{quotation}
\begin{quotation}
\begin{quotation}
pjon.\-set\-\_\-packet\-\_\-id(\-True).packet\-\_\-overhead()

\end{quotation}


\end{quotation}


\end{quotation}
11 \begin{quotation}
\begin{quotation}
\begin{quotation}
pjon.\-set\-\_\-synchronous\-\_\-acknowledge(\-True).packet\-\_\-overhead()

\end{quotation}


\end{quotation}


\end{quotation}
11 \begin{quotation}
\begin{quotation}
\begin{quotation}
pjon.\-set\-\_\-packet\-\_\-id(\-False).set\-\_\-asynchronous\-\_\-acknowledge(\-False).packet\-\_\-overhead()

\end{quotation}


\end{quotation}


\end{quotation}
9 \begin{quotation}
\begin{quotation}
\begin{quotation}
pjon.\-set\-\_\-crc\-\_\-32(\-False).include\-\_\-sender\-\_\-info(\-False).packet\-\_\-overhead()

\end{quotation}


\end{quotation}


\end{quotation}
5

``` 