Meant to replace all 1wire stuff (Dallas/\-Maxim commercial) some day. Meantwhile both busses/systems can be used together. The Software\-Bit\-Bang (S\-W\-B\-B) part is the O\-W\-P\-J\-O\-N or 1w\-P\-J\-O\-N bus.

``` 

 $\vert$ kubuntu 14.\-04 $\vert$ $\vert$ Arduino Uno $\vert$ $\vert$ Arduino Yun $\vert$ $\vert$ Through\-Serial $\vert$ $<$---$>$ $\vert$ Through\-Serial $\vert$ $\vert$ $\vert$ $\vert$ $\vert$ $\vert$ S\-W\-B\-B pin 7 $\vert$ $<$---$>$ $\vert$ S\-W\-B\-B pin 12 $\vert$ $\vert$ Server $\vert$ $\vert$ Master $\vert$ $\vert$ Slaves $\vert$ 



Server\-: L\-I\-N\-U\-X/\-Local/\-Through\-Serial/\-Remote\-Worker/\-Transmitter/ (kubuntu 14.\-04, T\-S) Master\-: A\-R\-D\-U\-I\-N\-O/\-Local/\-Through\-Serial/\-Software\-Bit\-Bang\-Surrogate/\-Surrogate/ (Uno, T\-S, S\-W\-B\-B pin 7) Slaves\-: A\-R\-D\-U\-I\-N\-O/\-Local/\-Software\-Bit\-Bang/\-Blink\-With\-Response/\-Receiver/ (Yun, S\-W\-B\-B pin12) ```

Notes\-:
\begin{DoxyItemize}
\item putting {\ttfamily bus.\-set\-\_\-synchronous\-\_\-acknowledge(false);} before {\ttfamily bus.\-begin();} in both, the Receiver and the Transmitter makes T\-S run stable
\item the terms \char`\"{}\-Master\char`\"{} and \char`\"{}\-Slave\char`\"{} might be missleading, as basically communication between all devices in any direction is possible
\item \href{https://github.com/gioblu/PJON/tree/master/src/strategies/SoftwareBitBang}{\tt https\-://github.\-com/gioblu/\-P\-J\-O\-N/tree/master/src/strategies/\-Software\-Bit\-Bang} (pull-\/down resistor etc.)
\end{DoxyItemize}

Arduino Uno and Yun can basically be any kind of A\-V\-R/\-Arduino (as supported by P\-J\-O\-N), no special needs except for a serial-\/usb port on one of them.

Linux parts (server) are derived from Raspi (R\-P\-I) and therefore need wiring\-Pi in order to compile succesfully\-: \begin{DoxyVerb}$ git clone git://git.drogon.net/wiringPi
$ cd wiringPi
$ ./build

$ make
$ sudo ./Transmitter
\end{DoxyVerb}


This setup can be used for developping, testing and also as a first productive system.

There are already solutions around like P\-J\-O\-N-\/g\-R\-P\-C or Module\-Interface\-: \href{https://github.com/fredilarsen/ModuleInterface/tree/master/examples/WebPage}{\tt https\-://github.\-com/fredilarsen/\-Module\-Interface/tree/master/examples/\-Web\-Page} Currently both are hard to compile and understand, but that will change (P\-J\-O\-N-\/g\-R\-P\-C is easier to compile in ubuntu 16.\-10, docu of Module\-Interface is up to be improved).

``` 

 $\vert$ kubuntu 14.\-04 $\vert$ $\vert$ Arduino Uno $\vert$ $\vert$ Arduino Yun $\vert$ $\vert$ Through\-Serial $\vert$ $<$---$>$ $\vert$ Through\-Serial $\vert$ $\vert$ $\vert$ $\vert$ $\vert$ $\vert$ S\-W\-B\-B pin 7 $\vert$ $<$---$>$ $\vert$ S\-W\-B\-B pin 12 $\vert$ $\vert$ Server $\vert$ $\vert$ Master $\vert$ $\vert$ Slaves $\vert$ 



Server\-: L\-I\-N\-U\-X/\-Local/\-Through\-Serial/\-Remote\-Worker/\-Device\-Generic/ (kubuntu 14.\-04, T\-S) Master\-: A\-R\-D\-U\-I\-N\-O/\-Local/\-Through\-Serial/\-Software\-Bit\-Bang\-Surrogate/\-Surrogate/ (Uno, T\-S, S\-W\-B\-B pin 7) Slaves\-: A\-R\-D\-U\-I\-N\-O/\-Local/\-Software\-Bit\-Bang/\-Device\-Generic/ A\-R\-D\-U\-I\-N\-O/\-Local/\-Software\-Bit\-Bang/\-O\-W\-P\-\_\-\-D\-G\-\_\-\-L\-C\-D\-\_\-\-Sensors/ A\-T\-T\-I\-N\-Y/\-Local/\-Software\-Bit\-Bang/\-Device\-Generic/ (Yun, S\-W\-B\-B pin12) ```

Notes\-:
\begin{DoxyItemize}
\item the attiny code uses 5554 bytes (67\%) compiled (may be less with L\-T\-O), see \href{https://github.com/drtrigon/sketchbook/blob/result/docu/OWPJON/ATTINY/Local/SoftwareBitBang/DeviceGeneric/DeviceGeneric.ino.compile#L55}{\tt https\-://github.\-com/drtrigon/sketchbook/blob/result/docu/\-O\-W\-P\-J\-O\-N/\-A\-T\-T\-I\-N\-Y/\-Local/\-Software\-Bit\-Bang/\-Device\-Generic/\-Device\-Generic.\-ino.\-compile\#\-L55}
\end{DoxyItemize}

The program {\ttfamily owpshell} (\hyperlink{namespaceDeviceGeneric}{Device\-Generic}) allows e.\-g. shell/command line access to the bus (similar to owshell read and write funcs). That can also be used from collectd e.\-g. It supports passing of data as command line parameter and via stdin (pipe). Later supports all posible chars.

Read device info\-: \begin{DoxyVerb}$ printf "\x01" | sudo ./owpshell /dev/ttyACM0 9600 44
\end{DoxyVerb}


Write to device memory and read for verification\-: \begin{DoxyVerb}$ printf "\x22ABC" | sudo ./owpshell /dev/ttyACM0 9600 44 | python unpack.py B
$ printf "\x21" | sudo ./owpshell /dev/ttyACM0 9600 44
\end{DoxyVerb}


Write a float e.\-g. for calibration/configuration of device\-: \begin{DoxyVerb}$ printf "\x32`python pack.py f 7.1`" | sudo ./owpshell /dev/ttyACM0 9600 44 | python unpack.py f
\end{DoxyVerb}


As can be seen the un-\//packing of values (casting from/to binary representation) is done by two python script {\ttfamily \hyperlink{unpack_8py}{unpack.\-py}} and {\ttfamily \hyperlink{pack_8py}{pack.\-py}}. Furthermore there is also a unittest python script called {\ttfamily test.\-py}.

Automatically detect the device type and test it\-: \begin{DoxyVerb}$ sudo python test.py
\end{DoxyVerb}


Force to test against a specific device type\-: \begin{DoxyVerb}$ sudo python test.py owp:dg:v1
\end{DoxyVerb}


Now the Through\-Serial (T\-S) part can be replaced by Local\-U\-D\-P (L\-U\-D\-P may be over wifi?) according to\-:

The setup can also be modified by replacing the Through\-Serial (T\-S) part by Local\-U\-D\-P (L\-U\-D\-P may be over wifi?). This needs an Ethernet\-Shield for the Arduino Uno used as Master.

``` 

 $\vert$ kubuntu 14.\-04 $\vert$ $\vert$ Arduino Uno $\vert$ $\vert$ Arduino Yun $\vert$ $\vert$ Local\-U\-D\-P $\vert$ $<$---$>$ $\vert$ Local\-U\-D\-P $\vert$ $\vert$ $\vert$ $\vert$ $\vert$ $\vert$ S\-W\-B\-B pin 7 $\vert$ $<$---$>$ $\vert$ S\-W\-B\-B pin 12 $\vert$ $\vert$ Server $\vert$ $\vert$ Master $\vert$ $\vert$ Slaves $\vert$ 



Server\-: L\-I\-N\-U\-X/\-Local/\-Local\-U\-D\-P/\-Remote\-Worker/\-Device\-Generic/ (kubuntu 14.\-04, L\-U\-D\-P) Master\-: A\-R\-D\-U\-I\-N\-O/\-Local/\-Software\-Bit\-Bang/\-Tunneler/\-Blinking\-Switch/ (Uno+\-Ethernet\-Shield W5100 based, L\-U\-D\-P, S\-W\-B\-B pin 7) Slaves\-: (all as mentioned before) ```

In this case the {\ttfamily owpshell} program works exactly the same except for the fact that it does not need any serial port info (and thus no sudo either)\-: \begin{DoxyVerb}$ printf "\x01" | ./owpshell - - 44
\end{DoxyVerb}


R\-A\-S\-P\-B\-E\-R\-R\-Y P\-I (1) S\-E\-T\-U\-P

We now want to setup this schemes (productive system)\-:

``` 

 $\vert$ linuxpc/raspi $\vert$ $\vert$ Arduino+\-E\-T\-H $\vert$ $\vert$ Arduino/\-A\-V\-R $\vert$ $\vert$ Local\-U\-D\-P $\vert$ $<$---$>$ $\vert$ Local\-U\-D\-P $\vert$ $\vert$ $\vert$ $\vert$ $\vert$ $\vert$ S\-W\-B\-B $\vert$ $<$---$>$ $\vert$ S\-W\-B\-B $\vert$ $\vert$ Server $\vert$ $\vert$ Master $\vert$ $\vert$ Slaves $\vert$ 

 ```

Installation of g++ 4.\-8 is needed if missing due to '-\/std=c++11' on Wheezy follow these steps\-:

``` pi $\sim$ \$ cat /etc/os-\/release ``` gives \char`\"{}wheezy\char`\"{}, see output\-: ``` P\-R\-E\-T\-T\-Y\-\_\-\-N\-A\-M\-E=\char`\"{}\-Raspbian G\-N\-U/\-Linux 7 (wheezy)\char`\"{} ... V\-E\-R\-S\-I\-O\-N=\char`\"{}7 (wheezy)\char`\"{} ``` so we can install g++ 4.\-8 in order to be able to compile P\-J\-O\-N 11.\-0 (adds $\sim$30-\/50\-M\-B) ``` pi $\sim$ \$ sudo apt-\/get install g++-\/4.8 pi $\sim$ \$ g++ ``` auto tab completition gives\-: g++ g++-\/4.6 g++-\/4.8

Next step is to get and compile (O\-W)P\-J\-O\-N 11\-:

``` pi $\sim$ \$ mkdir O\-W\-P\-J\-O\-N pi $\sim$ \$ cd O\-W\-P\-J\-O\-N/ pi $\sim$/\-O\-W\-P\-J\-O\-N \$ wget \href{https://github.com/gioblu/PJON/archive/11.0.tar.gz}{\tt https\-://github.\-com/gioblu/\-P\-J\-O\-N/archive/11.\-0.\-tar.\-gz} pi $\sim$/\-O\-W\-P\-J\-O\-N \$ tar -\/xvzf 11.\-0.\-tar.\-gz pi $\sim$/\-O\-W\-P\-J\-O\-N \$ cd P\-J\-O\-N-\/11.\-0/examples/\-L\-I\-N\-U\-X/\-Local/\-Local\-U\-D\-P/ pi $\sim$/\-O\-W\-P\-J\-O\-N/\-P\-J\-O\-N-\/11.0/examples/\-L\-I\-N\-U\-X/\-Local/\-Local\-U\-D\-P \$ mkdir -\/p Remote\-Worker/\-Device\-Generic/ pi $\sim$/\-O\-W\-P\-J\-O\-N/\-P\-J\-O\-N-\/11.0/examples/\-L\-I\-N\-U\-X/\-Local/\-Local\-U\-D\-P \$ cd Remote\-Worker/\-Device\-Generic/ ``` copy files, e.\-g. by copy-\/paste from kate to nano window\-: ``` pi $\sim$/\-O\-W\-P\-J\-O\-N/\-P\-J\-O\-N-\/11.0/examples/\-L\-I\-N\-U\-X/\-Local/\-Local\-U\-D\-P/\-Remote\-Worker/\-Device\-Generic \$ nano Makefile pi $\sim$/\-O\-W\-P\-J\-O\-N/\-P\-J\-O\-N-\/11.0/examples/\-L\-I\-N\-U\-X/\-Local/\-Local\-U\-D\-P/\-Remote\-Worker/\-Device\-Generic \$ nano Device\-Generic.\-cpp pi $\sim$/\-O\-W\-P\-J\-O\-N/\-P\-J\-O\-N-\/11.0/examples/\-L\-I\-N\-U\-X/\-Local/\-Local\-U\-D\-P/\-Remote\-Worker/\-Device\-Generic \$ nano \hyperlink{pack_8py}{pack.\-py} pi $\sim$/\-O\-W\-P\-J\-O\-N/\-P\-J\-O\-N-\/11.0/examples/\-L\-I\-N\-U\-X/\-Local/\-Local\-U\-D\-P/\-Remote\-Worker/\-Device\-Generic \$ nano \hyperlink{unpack_8py}{unpack.\-py} pi $\sim$/\-O\-W\-P\-J\-O\-N/\-P\-J\-O\-N-\/11.0/examples/\-L\-I\-N\-U\-X/\-Local/\-Local\-U\-D\-P/\-Remote\-Worker/\-Device\-Generic \$ chmod +x \hyperlink{pack_8py}{pack.\-py} \hyperlink{unpack_8py}{unpack.\-py} pi $\sim$/\-O\-W\-P\-J\-O\-N/\-P\-J\-O\-N-\/11.0/examples/\-L\-I\-N\-U\-X/\-Local/\-Local\-U\-D\-P/\-Remote\-Worker/\-Device\-Generic \$ ls -\/la total 72 drwxr-\/xr-\/x 2 pi pi 4096 Jul 7 15\-:06 . drwxr-\/xr-\/x 3 pi pi 4096 Jul 7 09\-:53 .. -\/rw-\/r--r-- 1 pi pi 4422 Jul 7 15\-:01 Device\-Generic.\-cpp -\/rw-\/r--r-- 1 pi pi 305 Jul 7 15\-:03 Makefile -\/rwxr-\/xr-\/x 1 pi pi 41259 Jul 7 15\-:04 owpshell -\/rwxr-\/xr-\/x 1 pi pi 162 Jul 7 15\-:06 \hyperlink{pack_8py}{pack.\-py} -\/rwxr-\/xr-\/x 1 pi pi 325 Jul 7 15\-:06 \hyperlink{unpack_8py}{unpack.\-py} pi $\sim$/\-O\-W\-P\-J\-O\-N/\-P\-J\-O\-N-\/11.0/examples/\-L\-I\-N\-U\-X/\-Local/\-Local\-U\-D\-P/\-Remote\-Worker/\-Device\-Generic \$ make raspi pi $\sim$/\-O\-W\-P\-J\-O\-N/\-P\-J\-O\-N-\/11.0/examples/\-L\-I\-N\-U\-X/\-Local/\-Local\-U\-D\-P/\-Remote\-Worker/\-Device\-Generic \$ printf \char`\"{}\textbackslash{}x01\char`\"{} $\vert$ ./owpshell -\/ -\/ 44 owp\-:dg\-:v1 pi $\sim$/\-O\-W\-P\-J\-O\-N/\-P\-J\-O\-N-\/11.0/examples/\-L\-I\-N\-U\-X/\-Local/\-Local\-U\-D\-P/\-Remote\-Worker/\-Device\-Generic \$ printf \char`\"{}\textbackslash{}x11\char`\"{} $\vert$ ./owpshell -\/ -\/ 44 $\vert$ ./unpack.py f 4.\-751999855041504, pi $\sim$/\-O\-W\-P\-J\-O\-N/\-P\-J\-O\-N-\/11.0/examples/\-L\-I\-N\-U\-X/\-Local/\-Local\-U\-D\-P/\-Remote\-Worker/\-Device\-Generic \$ printf \char`\"{}\textbackslash{}x12\char`\"{} $\vert$ ./owpshell -\/ -\/ 44 $\vert$ ./unpack.py f 28.\-02459144592285, ``` and finally add the sensors/devices to your logging system e.\-g. collectd\-: ``` \$ python owpjon\-\_\-sensor.\-py \$ collectd -\/\-C /etc/collectd/collectd.conf -\/\-T \$ cat /var/log/syslog \$ sudo /etc/init.d/collectd restart ```

Possible wireless solutions going through wall e.\-g. (galvanic isolated)\-: (would allow to turn wifi -\/ used for 1wire -\/ off during night again)

``` 

 $\vert$ Dragino\-L\-G01-\/\-S $\vert$ $\vert$ Arduino+\-Lo\-Ra $\vert$ $\vert$ Arduino/\-A\-V\-R $\vert$ $\vert$ Lo\-Ra $\vert$ $<$---$>$ $\vert$ Lo\-Ra $\vert$ $\vert$ $\vert$ $\vert$ S\-W\-B\-B $\vert$ $\vert$ S\-W\-B\-B $\vert$ $<$---$>$ $\vert$ S\-W\-B\-B $\vert$ $\vert$ Slaves/\-Tunnel $\vert$ $\vert$ Master/\-Tunnel $\vert$ $\vert$ Slaves $\vert$ 

 S\-W\-B\-B bus indoors S\-W\-B\-B bus outdoors ```

Dragino is 3v3 and needs a level shifter -\/ I used this circuit (somehow inspired by V-\/\-U\-S\-B but is not the same)\-: ``` 3v3 I\-O ---o---\textbackslash{}/\textbackslash{}/\textbackslash{}/-\/-- 5v I\-O \subsection*{$\vert$ resistor (68 better 150...4k7) }

\subsection*{$\vert$ $\vert$ 3v6 Zener (protect 3v3) }

$\vert$ G\-N\-D ``` see also \href{http://www.partsim.com/simulator/#148247}{\tt http\-://www.\-partsim.\-com/simulator/\#148247}.

It works but has at least these disadvantages\-:
\begin{DoxyItemize}
\item knee of zener known not to be very sharp, so 3v6 are not precise
\item using 150 ohms as minimum is recommended in order to be protected against shorts
\begin{DoxyItemize}
\item a device on the 5v bus needs to be able to drive (5v -\/ 3.\-6v) / R $\sim$ 0.\-4..20m\-A for a high
\item currently 4k7 is used
\item current through zener might peak if 3.\-3\-V supply is higher (simulation above shows about 3.\-6\-V@20m\-A -\/ so using 3.\-6\-V zener might be the safe choice)
\end{DoxyItemize}
\item (cut-\/off frequency might be low due to big capacity and big resistor)
\end{DoxyItemize}

Further improvements might include; setup a \char`\"{}3 way\char`\"{} tunneler (L\-U\-D\-P, S\-W\-B\-B, S\-W\-B\-B) that gives 2 S\-W\-B\-B pins (on same bus; a 5v and a 3v3 bus).

An alternative proposal using a schottky diode is shown in \href{http://www.partsim.com/simulator/#148244}{\tt http\-://www.\-partsim.\-com/simulator/\#148244}.

Notes\-:
\begin{DoxyItemize}
\item O\-W\-P\-\_\-\-D\-G\-\_\-1w-\/adaptor for mega328 could may be modified to fit in an attiny (mega88); O\-W\-P\-J\-O\-N-\/1wire (master) converter for refurbishing of old 1wire devices
\begin{DoxyItemize}
\item \href{https://github.com/mikaelpatel/Arduino-OWI,}{\tt https\-://github.\-com/mikaelpatel/\-Arduino-\/\-O\-W\-I,} \href{https://github.com/mikaelpatel/Arduino-OWI/tree/master/examples/ATtiny}{\tt https\-://github.\-com/mikaelpatel/\-Arduino-\/\-O\-W\-I/tree/master/examples/\-A\-Ttiny}
\end{DoxyItemize}
\item Outdoor/\-Isolated bus branch; \href{https://github.com/gioblu/PJON/issues/222}{\tt https\-://github.\-com/gioblu/\-P\-J\-O\-N/issues/222}
\begin{DoxyItemize}
\item Lo\-Ra-\/\-S\-W\-B\-B switch/tunnel; Dargino/\-Yun for Lo\-Ra indoors, Uno+\-Lo\-Ra outdoors -\/ outdoor low-\/power possible
\item alternative solutions;
\begin{DoxyItemize}
\item Analog\-Sampling-\/\-S\-W\-B\-B\-: Single L\-E\-D or Laser Diode bidirectional through window -\/ low-\/power posible -\/ needs testing and safety measure, e.\-g. cats remove sender and look into laser diode ...
\item Over\-Sampling-\/\-S\-W\-B\-B\-: 315/433\-M\-Hz or H\-C-\/12 radio -\/ low-\/power possible? where to order from?
\item L\-U\-D\-P-\/\-S\-W\-B\-B (L\-U\-D\-P via wifi); Yun, E\-S\-P8266 (G\-U\-P\-D, 3.\-3v), raspi zero (3.\-3v using wifi stick or zero w), uno+eth+lan2wifi (like now), maybe use yun as \char`\"{}router\char`\"{}/lan2wifi (\href{https://hackingmajenkoblog.wordpress.com/2017/06/02/configuring-yun-wifi/}{\tt https\-://hackingmajenkoblog.\-wordpress.\-com/2017/06/02/configuring-\/yun-\/wifi/}) -\/ wifi thus high-\/power
\end{DoxyItemize}
\item for testing and specifing network stability use; \href{https://github.com/gioblu/PJON/tree/master/examples/ARDUINO/Local/Any/StrategyLinkNetworkAnalysis}{\tt https\-://github.\-com/gioblu/\-P\-J\-O\-N/tree/master/examples/\-A\-R\-D\-U\-I\-N\-O/\-Local/\-Any/\-Strategy\-Link\-Network\-Analysis} (same for Software\-Bit\-Bang, Analog\-Sampling)
\end{DoxyItemize}
\item Setup, test and enjoy \href{https://github.com/fredilarsen/ModuleInterface/tree/master/examples/WebPage}{\tt https\-://github.\-com/fredilarsen/\-Module\-Interface/tree/master/examples/\-Web\-Page} (may be on a Raspi2)
\item Arduino+\-E\-T\-H can be replaced by E\-S\-P8266 (3.\-3$<$-\/$>$5.\-0v shifting might be needed, may be not) but needs Global\-U\-D\-P (L\-U\-D\-P would be nice)
\item Arduino+\-E\-T\-H could be replaced by Arduino Yun or even Dragino, if P\-J\-O\-N L\-U\-D\-P would support Yun (needs Yun to support U\-D\-P)
\item Server and Master might be joined (linuxpc/raspi + Arduino+\-E\-T\-H) into a Raspi(\-Zero) with wifi (3.\-3$<$-\/$>$5.\-0v shifting?)
\item Local\-U\-D\-P could also be replaced by Ethernet\-T\-C\-P but like Global\-U\-D\-P not so convenient 
\end{DoxyItemize}